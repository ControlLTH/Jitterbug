% Dokumentdatablad f�r rapporter
% ==============================
% Uppgifterna fylls i p� engelska inom {}.
% De prickade raderna skall alltid fyllas i (och �verfl�diga prickar tas bort),
% �vriga uppgifter �r beroende av rapportens art.
%
% \name          Ex: FINAL REPORT , MASTER THESIS , INTERNAL REPORT
% \date          M�nad och �rtal, ex: October 1985
% \num           Nummer (av Britt-Marie), dvs endast siffrorna.
% \author        Namn p� f�rfattare
% \supervisor    Handledare
% \title         Rapportens titel (ev med engelsk �vers.)
% \keywords      Eventuella nyckelord
% \language      Det spr�k rapporten �r skriven p�
% \pages         Totalt antal sidor
% \begin{abstract}...\end{abstract} Abstract p� engelska
%
%------------------------------------------------------------------------------
%
\documentclass{docdata}
\usepackage{schoolbook}
\begin{document}
\name{INTERNAL REPORT}
\date{January 2003}
\num{7604}
\author{Anton Cervin, Bo Lincoln}
\supervisor{}
\sponsor{}
\title{Jitterbug Reference Manual}
\keywords{}
\classification{}
\supplement{}
\isbn{}
\language{English}
\pages{37}
\security{}
\recipient{}
\begin{abstract}
The manual describes the use of Jitterbug, a Matlab toolbox for
real-time control performance analysis. The tool facilitates the
computation of a quadratic performance index for a linear control
system under various timing conditions.
\end{abstract}
\end{document}
